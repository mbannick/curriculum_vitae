%%%%%%%%%%%%%%%%%%%%%%%%%%%%%%%%%%%%%%%%%
% "ModernCV" CV and Cover Letter
% LaTeX Template
% Version 1.11 (19/6/14)
%
% This template has been downloaded from:
% http://www.LaTeXTemplates.com
%
% Original author:
% Xavier Danaux (xdanaux@gmail.com)
%
% License:
% CC BY-NC-SA 3.0 (http://creativecommons.org/licenses/by-nc-sa/3.0/)
%
% Important note:
% This template requires the moderncv.cls and .sty files to be in the same 
% directory as this .tex file. These files provide the resume style and themes 
% used for structuring the document.
%
%%%%%%%%%%%%%%%%%%%%%%%%%%%%%%%%%%%%%%%%%

%----------------------------------------------------------------------------------------
%	PACKAGES AND OTHER DOCUMENT CONFIGURATIONS
%----------------------------------------------------------------------------------------

\documentclass[11pt,letterpaper,sans]{moderncv} % Font sizes: 10, 11, or 12; paper sizes: a4paper, letterpaper, a5paper, legalpaper, executivepaper or landscape; font families: sans or roman

\moderncvstyle{classic}
\moderncvcolor{husky} % CV color - options include: 'blue' (default), 'orange', 'green', 'red', 'purple', 'grey' and 'black'

\usepackage{lipsum} % Used for inserting dummy 'Lorem ipsum' text into the template
\usepackage{fontawesome}
\usepackage{hyperref}
\hypersetup{colorlinks,breaklinks,
            urlcolor=[RGB]{130, 96, 165},
            linkcolor=[RGB]{130, 96, 165}}

\usepackage[scale=0.85]{geometry} % Reduce document margins
%\setlength{\hintscolumnwidth}{3cm} % Uncomment to change the width of the dates column
%\setlength{\makecvtitlenamewidth}{10cm} % For the 'classic' style, uncomment to adjust the width of the space allocated to your name

%----------------------------------------------------------------------------------------
%	NAME AND CONTACT INFORMATION SECTION
%----------------------------------------------------------------------------------------
\usepackage{tikz}
\usepackage{doi}
\usepackage{mathptmx}
\usepackage[11pt]{moresize}
\usepackage{multicol}

\usepackage[
  backend=biber,
  defernumbers=true,
  sorting=none,
  maxbibnames=20,maxcitenames=20,
  style=nejm,
  articledoi=true
]{biblatex}
\addbibresource{refs.bib}

\makeatletter
\newcommand{\citesinthissection}[1]{\xdef\@totalcites{#1}}
\newcounter{numbibentries}
\renewbibmacro*{finentry}{\stepcounter{numbibentries}\finentry}
\defbibenvironment{bibliography}
  {\list
     {\printtext[labelnumberwidth]{% label format from numeric.bbx
        \printfield{labelprefix}%
        \abx@field@labelnumber\relax}}
     {\setlength{\topsep}{0pt}
      \setlength{\labelwidth}{\hintscolumnwidth}%
      \setlength{\labelsep}{\separatorcolumnwidth}%
      \leftmargin\labelwidth%
      \advance\leftmargin\labelsep}%
      \sloppy\clubpenalty4000\widowpenalty4000}
  {\endlist}
  {\item}
\makeatother


\firstname{Marlena} % Your first name
\familyname{Bannick} % Your last name

\mobile{+1-206-498-0046}
\email{mnorwood@uw.edu}

\begin{document}

\makecvtitle

\section{Education}

\cventry{2019--present}{Doctor of Philosophy in Biostatistics}{University of Washington}{}{}{}

\cventry{2016--2019}{Master of Science in Biostatistics}{University of Washington}{}{}{
Committee: Ruth Etzioni chair; Megan Othus \\ 
Thesis: Estimating time to intermediate endpoints using population-level survival data and deconvolution methods, with application to cancer progression and recurrence
}

\cventry{2012--2016}{Bachelor of Science in Public Health}{University of Washington}{}{}{
    Minor in Mathematics; College Honors \\
    \textit{magna cum laude}; Phi Beta Kappa
}

\section{Experience}

\cventry{2019--present}{Mathematical Sciences Researcher}{Institute for Health Metrics and Evaluation}{}{}{
Supervisor: Dr. Aleksandr Aravkin
\vspace{1mm}
\begin{itemize}
    \item Develop new quantitative methods and modeling strategies that incorporate all possible relevant global health data to achieve credible and policy-relevant results
    \item \href{https://github.com/ihmeuw-msca}{Implement new and existing quantitative methods into code} that address analytical challenges experienced across teams at IHME
    \item Lead Python software developer for a \href{https://github.com/ihmeuw/cascade-at}{multi-stage hierarchical disease dynamics model}
    \item Translate quantitative methodology to IHME research teams through consultations
    \item Designed and built \href{https://github.com/ihmeuw/covid-model-seiir-pipeline}{COVID-19 modeling software} for the \href{https://covid19.healthdata.org/projections}{IHME COVID-19 projections}
\end{itemize}
}

\cventry{2016--2019}{Post Bachelor Fellow}{Institute for Health Metrics and Evaluation}{}{}{
Supervisors: Dr. Stephen Lim, Dr. Kyle Foreman, Dr. Theo Vos
\vspace{1mm} \\
\underline{Central Computation for the \href{http://www.healthdata.org/gbd}{Global Burden of Disease Study}}
\vspace{1mm}
\begin{itemize}
    \item Backend development for the institutional statistical modeling program to make \href{https://vizhub.healthdata.org/cod/}{ cause of death estimates} for the Global Burden of Disease Study 2017 and 2019
    \item Designed software for a large cluster computing platform used by dozens of disease modelers as a key part of the GBD estimation pipeline
\end{itemize}
\vspace{1mm} \\
\underline{Natural Language Processing Applications}
\vspace{1mm}
\begin{itemize}
    \item Developed a tool to screen the results of PubMed queries for relevance to research teams at IHME using natural language processing and deep learning methods
\end{itemize}
\vspace{1mm} \\
\underline{Disease Estimation for the \href{http://www.healthdata.org/gbd}{Global Burden of Disease Study}}
\vspace{1mm}
\begin{itemize}
    \item Developed estimates of \href{http://ihmeuw.org/56tv}{non-fatal injury burden} for the Global Burden of Disease Study 2016
    \item Conceptualized and implemented strategies to estimate \href{http://ihmeuw.org/56tu}{sexual violence indicators} for the Sustainable Development Goals
\end{itemize}
}

\cventry{2015--2016}{Research Assistant}{Fred Hutchinson Cancer Research Center}{}{}{
Supervisor: Dr. Beth Mueller
\vspace{1mm}
\begin{itemize}
    \item Performed statistical analyses for a cohort study of pregnancy outcomes in women with multiple sclerosis, and a \href{https://doi.org/10.1371/journal.pone.0179006}{case-control study of congenital malformations and childhood cancer}
    \item Researched the capacity of each state in the U.S. to link birth certificates to state cancer registries for a National Cancer Institute-funded study
\end{itemize}
}

\cventry{06--09/2015}{Research Assistant}{Department of Biostatistics, University of Washington}{}{}{
Supervisor: Dr. James Hughes
\vspace{1mm}
\begin{itemize}
    \item Developed a statistical method to estimate under-reporting of sensitive, self-reported behaviors in a study population with biomarkers
    \item \href{https://doi.org/10.1016/j.annepidem.2016.07.011}{Authored a publication on the novel method} that was ultimately presented by Dr. Hughes at the CDC Expert Consultation on Advancing Methods for Biobehavioral Surveys in 2018
\end{itemize}
}

\cventry{06--08/2014}{Research Assistant}{Fred Hutchinson Cancer Research Center}{}{}{
Supervisor: Dr. Deborah Donnell, HIV Prevention Trials Network
\vspace{1mm}
\begin{itemize}
    \item Developed an R program for an HIV Prevention Trials Network study to inform the categorization of biological specimens in a way that optimized sensitivity and specificity
\end{itemize}
}

\cventry{04--08/2014}{Student Research Assistant}{University of Washington}{}{}{
Center for Clinical and Epidemiological Research
\vspace{1mm}
\begin{itemize}
    \item Supported the maintenance of a large health research registry
    \item Performed targeted literature reviews to inform grants for new epidemiological twin studies
\end{itemize}
}

\cventry{2013--2014}{Student Research Assistant}{University of Washington}{}{}{
Supervisor: Dr. Suzanne Kerns, Division of Public Behavioral Health and Justice Policy
\vspace{1mm}
\begin{itemize}
    \item Analyzed qualitative survey data using ATLAS.ti to determine barriers to implementing evidence-based parenting interventions in Washington State
    \item Designed online data collection platforms for intervention monitoring and evaluation
    \item Assisted in writing monitoring and evaluation progress reports for the Washington State Division of Behavioral Health and Recovery
\end{itemize}
}

\section{Honors and Awards}

\cvitem{2019}{\href{https://www.biostat.washington.edu/news/congratulations-our-2019-biostatistics-graduates}{Senior MS in Biostatistics Award}, Department of Biostatistics, University of Washington}
\cvitem{2019}{Graduate School Conference Travel Award, University of Washington}
\cvitem{2016}{\href{https://www.washington.edu/husky100/year/2016/#name=marlena-norwood}{Husky 100 Award}, University of Washington}
\cvitem{2016}{\href{https://sph.washington.edu/about/sph-awards-archive}{Outstanding Student Award}, School of Public Health, University of Washington}

\section{Publications}
\nocite{*}
\printbibliography[type=article,heading=none, notkeyword={GBD}]

\vspace{0.25cm}
\subsection{Global Burden of Disease Collaboration}
\cvitem{}{Performed analyses and developed statistical and computational machinery for the \href{https://www.thelancet.com/gbd}{Global Burden of Disease Study}, significantly contributing to the following publications.}
\vspace{0.10cm}
\nocite{*}
\printbibliography[type=article, heading=none, keyword={GBD}]

\section{Presentations}

\nocite{*}
\printbibliography[type=misc,heading=none]

\section{Teaching}

\subsection{Guest Lectures}

\cvitem{07/2020}{"Introduction to Epidemiological and Biostatistical Thinking", Neurology Clinical Fellowship Didactics \textit{Instructor: Dr. Andrea Cheng-Hakimian}, University of Washington, Seattle. Materials: \href{https://github.com/mbannick/uw-neurology-fellows}{https://github.com/mbannick/uw-neurology-fellows}}

\cvitem{08/2018 \& 2019}{"Cause of Death Ensemble Model (CODEm)", Global Burden of Disease (GH 590) \textit{Instructor: Dr. Jeffrey Stanaway}, Department of Global Health, University of Washington, Seattle}.

\subsection{Workshops}

\cvitem{05/2019}{"Data to DALYs: Case Study on Diabetes", with Dr. Theo Vos and Dr. Liane Ong. 2-day short course. Global Burden of Disease Workshop, Eretria, Greece.}

\section{Service and Affiliations}

\subsection{Affiliations}
\cvitem{2019--present}{American Statistical Association}

\subsection{Peer Review}
\cvitem{2020}{Journal of Medical Internet Research}

\section{Additional Training}
{\small
\renewcommand{\listitemsymbol}{-~} % Changes the symbol used for lists

\cvitem{2015}{Summer Institute in Statistics and Modeling in Infectious Diseases, University of Washington}

\cvitem{2015}{Writing in the Sciences, \textit{with distinction}, via Stanford Online, Lagunita}

\par}

\end{document}



